%================================================================
%                 B E S S E L  F U N C T I O N S
%================================================================
%                             NOTES
%                           ---------
%
%                           ---------
%================================================================
\section{Bessel functions}
%
Bessel functions are solutions to the Bessel differential equation.
%
  \begin{defn}The Bessel differential equation of order $\nu$ and argument $z$
  %
    \begin{equation}\label{eq:bessel_differential}
        z^2 \frac{d^2 u}{dz^2} + z\frac{du}{dz} + (z^2 - \nu^2)u = 0
    \end{equation}
  %
  for $\nu, z \in \bb{C}$.
  %
  \end{defn}\par
%
  \begin{defn}\label{defn:bessel_func}
  Bessel functions, or cylindrical functions of the first kind, are solutions to the Bessel differential equation, \eqref{eq:bessel_differential}.
  %
    \begin{align*}
        J_\nu(z) = \sum^\infty_{m=0} \frac{
        (-1)^m (z/2)^{2m+\nu} }{
        m! (\nu+m)! }
    \end{align*}
  %
  \end{defn}
%
For $\nu$ not an integer, the collection of Bessel functions of different $\nu$ is linearly independent and is the general soltion to \ref{eq:bessel_differential}. However if $\nu = n \in \bb{Z}$ Bessel functions are linearly dependent. In particular, the following identity holds.
%
  \begin{propn} \label{propn:bessel_int_order_identity} For $n \in \bb{Z}$,
    \begin{equation}
      J_n(z) = (-1)^n J_{-n}(z).
    \end{equation}

  \end{propn}\par
%
This property is stated in all sources regarding Cylindrical functions, for example see \cite{culham04bessel, karmazina19cylinderfunc, korenev02bessel_func}.\par
%
Hence for $n$ an integer the Bessel functions no longer form the solution to the Bessel differential equation, so we introduce cylindrical functions of the second kind: Neumann functions.
%
  \begin{defn}\label{defn:neumann_func}
    Neumann functions, are linear combinations of Bessel functions of the first kind.
    %
      \begin{gather*}
          Y_\nu (z) = \frac{J_\nu(z) \cos (\nu \pi) - J_{-\nu}(z)}{\sin (\nu \pi)} \text{ for } \nu \notin \bb{Z}\\
          Y_n = \lim_{\nu \rightarrow n} Y_\nu (z) \text{ for } n \in \bb{Z}
      \end{gather*}
    %
  \end{defn}\par
%                                         DEFN: HANKEL
  \begin{defn}\label{defn:hankel_func}
    Cylindrical functions of the third kind or Hankel functions, are linear combinations of Bessel functions of the first and second kind.
      \begin{align*}
          H^{(1)}_\nu(z) = J_\nu(z) + i Y_\nu(z)\\
          H^{(2)}_\nu(z) = J_\nu(z) - i Y_\nu(z)
      \end{align*}
    In particular, for $\nu = n \in \bb{Z}$,
      \begin{align*}
        H^{(1)}_\nu(z)
          &= \lim_{\nu \rightarrow n} \frac{J_{\nu}(z) - e^{-\nu\pi i} J_\nu(z)}{i \sin(\nu \pi)} \\
        H^{(2)}_\nu(z)
          &= \lim_{\nu \rightarrow n} \frac{J_{\nu}(z) - e^{\nu\pi i} J_\nu(z)}{- i \sin(\nu \pi)}
      \end{align*}
    from \cite{karmazina19cylinderfunc}.
  \end{defn}
