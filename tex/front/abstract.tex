The aim of this project is to better understand the scattering of acoustic waves around different objects. The starting point for this are the Navier-Stokes equations and a simplified physical model of acoustic wave motion in air. Although in this project we largely consider the wave motion to be an acoustic wave travelling through air, the scattering of a wave incident on an obstacle has a wide scope of application to many other fields, like electromagnetism, seismology and hydrodynamics.


The model of wave motion relies on assumptions about the nature of fluid flows, namely, that air is a barotropic, adiabatic ideal gas. We use separation of variables and perturbation theory to arrive at the linear wave equation, and then the Helmholtz equation, to finally encounter the Bessel differential equation. Bessel equations form the basis of this project.

We solve two related problems. First we considered the field generated by a plane wave inciding on a cylinder with two types of boundary conditions: Neumann and Dirichlet. Finally, we allow transmission through the boundary of the cylinder and consider the field generated in that case.

The last chapter is concerned with the tool I created to plot the solution to this problems. Although not all together succesful, it was an interesting aspect of this project. 
