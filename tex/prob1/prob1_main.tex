%================================================================
%                           N O T E S
%                           ---------
%
%                           ---------
%-----------------------------------------------------------------
%                       INTRODUCTION
%-----------------------------------------------------------------
\section{Introduction to the problem}
We consider a plane wave propagating from infinity onto a cylinder centered at the origin and of radius $\sigma$, as depicted in \figref{fig:problem_1}. Our goal is to find an expression for the total field resulting from the interaction of the incident wave with the cylinder. We consider two types of conditions at the boundary, Neumann and Dirichlet, and create plots in python for these. As showed in the first chapter, it will be sufficient to seek $\Phi(x, y)$ to determine the velocity field.\par
%
    \begin{figure}
        \centering
        \includegraphics[width=6cm]{../figures/sk_problem_1.png}
        \caption{Problem 1}
        \label{fig:problem_1}
    \end{figure}
%
Let $\Phitot = \Phiin + \Phisc$, where $\Phiin$ is the incident field, and $\Phisc$ is the scattered field. All three of these must satisfy the Helmholtz equation.
%====================================================================
%                   INCIDENT FIELD
%====================================================================
\section{The incident field}
First let us consider $\Phiin$, the incident field. We choose Cartesian coordinates $(x,y)$, and polar coordinates $(r, \theta)$.
    \begin{propn}\label{propn:ch2_inc_field_intro}
    The incident field has the form
        \begin{equation}
            \Phiin = e^{-i(\vec{k}\cdot\vec{x})}
        \end{equation}
    \end{propn}
    \begin{proof}
    We need to show that this expression for $\Phiin$ satisfies the Helmholtz equation, \eqref{eq:helmholtz}. \par
        \begin{align*}
            \nabla^2 \Phiin
            &= \left[ \partialfrac{^2}{x^2} + \partialfrac{^2}{y^2} \right] e^{-i(\vec{k}\cdot\vec{x})} \\
            &= (-i)^2 a^2 e^{-i(ax+by)} + (-i)^2b^2 e^{-i(ax+by)}\\
            &= -(a^2+b^2)\Phiin
        \end{align*}
    Then,
        \begin{gather*}
            \nabla^2 \Phiin + k^2 \Phiin
            = -(a^2+b^2)\Phiin + k^2 \Phiin = 0\\
            \Rightarrow k^2-(a^2+b^2)=0
        \end{gather*}
    which is true by definition of the wavevector (definition \ref{defn:wave_vector}).
    \end{proof}

In the first chapter we define the wave vector $\vec{k}$ as $(k\cos\alpha, \sin\alpha)$, where $\alpha$ is the incident angle, and $k$ is the wavenumber. We can set $\alpha = 0$ without loss of generality by adjusting the coordinate system accordingly. Hence we have $\vec{k}=(k, 0)$.

\begin{propn}\label{propn:ch2_inc_field_infty_sum}
The incident field can be expressed as an infinite sum as follows.
    \begin{equation*}\label{eq:ch2_inc_field_infty_sum}
        \Phiin = \sum^{\infty}_{n=0} \epsilon_n i^n J_n(kr) \cos (n\theta)
    \end{equation*}
\end{propn}
\begin{proof} From the Jacobi expansion, proposition \ref{propn:jacobi_expansion} and proposition \ref{propn:ch2_inc_field_intro} we have
  \[\Phiin =e^{ikr\cos\theta} = \sum^{\infty}_{n=0} \epsilon_n i^n J_n(kr) \cos(n\theta) \].
\end{proof}
%-----------------------------------------------------------------
%                       SCATTERED FIELD
%-----------------------------------------------------------------
\section{The scattered field}\label{ss:scattered_field}
To find the expression for the scattered field we will again use the method of separation of variables. We make two assumptions for the potential $\Phisc$:
\begin{enumerate}[label=(\roman*)]
  \item it is separable into two indepdent functions, one for $r$ and one for $\theta$, and
  \item it satisfies the Helmholtz equation.
\end{enumerate}
%
\begin{propn}\label{propn:scattered_field_differential_equations}
  With these two assumptions, we get two independent differential equations to solve to find the radial and azimuthal components of the velocity field.
  %
  \begin{equation}\label{eq:ch2_theta_dep}
      \frac{d^2 \Theta}{d\theta^2} + \hat{\nu} \Theta = 0, ~\text{and}
  \end{equation}
  %
  \begin{equation}\label{eq:ch2_r_dep}
      r^2 \frac{d^2 R}{dr^2} + r \frac{d R}{dr} + (k^2r^2 - \hat{\nu}) R = 0.
  \end{equation}
  %
  Where $\hat{\nu} \in \bb{C}$.
\end{propn}
%
\begin{proof}
  From assumption (i) we have
  \begin{equation}
      \Phisc = R(r) \Theta(\theta),
  \end{equation}\par
  %
  and from (ii)
  \begin{equation}\label{eq:phisc_helmholtz}
      \nabla^2 \Phisc + k^2 \Phisc = 0.
  \end{equation}
  %
  Hence,
  \begin{align*}
      \frac{1}{r} \partialfrac{}{r}
      \left( r \partialfrac{}{r} [R(r) \Theta(\theta)] \right)
      &+ \frac{1}{r^2} \partialfrac{^2}{\theta^2} [R(r) \Theta(\theta)]
      + k^2 R(r) \Theta(\theta) = 0 \\
      %
      \Theta(\theta) \left( \frac{1}{r} \frac{dR(r)}{dr} + \frac{d^2 R(r)}{dr^2} \right)
      &+ \frac{R(r)}{r^2} \left( \frac{d^2 \Theta(\theta)}{d\theta^2}\right)
      + k^2 R(r) \Theta(\theta) = 0 \\
      %
      \Theta \left( \frac{1}{r} R' + R''\right)
      &+ \frac{1}{r^2} R \Theta'' + k^2 R \Theta = 0
  \end{align*}
  \begin{equation*}
       r^2 \frac{R''}{R} + r\frac{R'}{R} + (kr)^2 = - \frac{\Theta''}{\Theta}
  \end{equation*}\par
  %
  For this to be true we must have the left hand side and the right hand side equal to the same constant, $\hat{\nu}$. That is,
  \begin{gather*}
    r^2 \frac{R''}{R} + r\frac{R'}{R} + (kr)^2 = \hat{\nu} \\
    %
    \text{and } \frac{\Theta''}{\Theta} = - \hat{\nu}
  \end{gather*}
  as required.
\end{proof}

%-------------------- principle of superposition ---------------------

\begin{thm}The principle of superposition for second order homogeneous linear equations is that statement that if $y_1$ and $y_2$ are any two solutions to the equation
  \[ \ddot{y} + p(t)\dot{y} + q(t)y =  0\]
  then any function of the form
  $ y_0 = C_1 y_1 + C_2 y_2 $
  is also a solution of the equation.
\end{thm}
\begin{proof}
  Consider the function $y_0$ as defined above. Then
  \begin{gather*}
    \dot{y}_0 = C_1 \dot{y}_1 + C_2 \dot{y}_2 \\
    \ddot{y}_0 = C_1 \ddot{y}_1 + C_2 \ddot{y}_2.
  \end{gather*}
  So the ODE becomes
  \begin{gather*}
    C_1 \ddot{y}_1 + C_2 \ddot{y}_2 + p(t) [ C_1 \dot{y}_1 + C_2 \dot{y}_2 ] + q(t) [C_1 y_1 + C_2 y_2] = 0
  \end{gather*}
  which holds if and only if
  \begin{gather*}
    \left\{
    \begin{array}{l}
      C_1[\ddot{y_1} + p(t)\dot{y_1} + q(t)y_1] =  0 ~~\text{and}\\
      C_2[\ddot{y_2} + p(t)\dot{y_2} + q(t)y_2] =  0.
    \end{array}
    \right.
  \end{gather*}
  By definition, $y_1$ and $y_2$ are solutions to the ODE, therefore $ y_0 = C_1 y_1 + C_2 y_2 $ is also a solution for any constants $C_1$, $C_2$ as required.
\end{proof}
%=====================================================================
%           propn: general solution for angular dependence
%=====================================================================
\begin{propn}
  The expression
  \[
  \Theta(\theta) = A \cos(n\theta) + B \sin(n\theta)
  \]
  solves \eqref{eq:ch2_theta_dep}.
\end{propn}
\begin{proof}
  Let $\hat{\nu} = \nu^2 \in \bb{C}$. Then we have the linear homogeneous differential equation
  \begin{equation*}
    \partialfrac{^2\Theta}{\theta^2} - \nu^2 \Theta = 0.
  \end{equation*}
  This has solution
  \begin{equation*}
    \Theta(\theta) = C_1 e^{\nu\theta} + C_2 e^{-\nu\theta}
  \end{equation*}
  Consider $\nu = \nu_1 + i\nu_2$. Then
  \begin{align*}
    \Theta(\theta) = C_1 e^{\nu_1\theta}e^{i\nu_2\theta} + C_2 e^{-\nu_1\theta}e^{-i\nu_2\theta}.
  \end{align*}
  But we expect our solution to be periodic in $\theta$, so we must have $\nu_1 = 0$. Hence $\nu^2 = -\nu_2^2 \leq 0$, and we have a trigonometric solution
  \begin{equation}
    \Theta(\theta) = A \cos(\nu\theta) + B \sin(\nu\theta).
  \end{equation}\par

  Since our solution must be $2\pi$ periodic, we have
  \begin{align*}
    \Theta(\theta) &= \Theta(\theta+2 \pi)\\
    A \cos(\nu\theta) + B \sin(\nu \theta) &= A \cos(\nu\theta+2\pi\nu) + B \sin(\nu \theta + 2 \pi \nu)
  \end{align*}
  %
  \begin{multline*}
      A \cos(\nu\theta) + B \sin(\nu \theta) \\
      = A [\cos(\nu\theta)\cos(2\pi\nu) - \sin(2\pi\nu)\sin(\nu\theta)] \\
      + B [\sin(\nu\theta)\cos(2\pi\nu) + \cos(\nu\theta)\sin(2\pi\nu)]
  \end{multline*}
  %
  \begin{multline*}
      \therefore A\cos(\nu\theta) + B \sin(\nu \theta) \\
      = \cos(\nu\theta)[A\cos(2\pi\nu) + B\sin(2\pi\nu)] \\
      + \sin(\nu\theta)[B\cos(2\pi\nu) - A\sin(2\pi\nu)]
  \end{multline*}\par
  %
  So we have
  \begin{align*}
    A &= A\cos(2\pi\nu) + B\sin(2\pi\nu) \text{ and}\\
    B &= B\cos(2\pi\nu) - A\sin(2\pi\nu).
  \end{align*}\par
  %
  Or equivalently,
  \begin{align}\label{eq:eigenvalue_proof_angular_dep}
    \begin{pmatrix}
      1 - \cos(2\pi\nu) & -\sin(2\pi\nu)\\
      \sin(2\pi\nu) & 1 - \cos(2\pi\nu)
    \end{pmatrix}
    \begin{pmatrix}
      A\\
      B
    \end{pmatrix}
    = 0
  \end{align}\par
  %
  We'll call the matrix on the left hand side $\mathbf{M}$. For \eqref{eq:eigenvalue_proof_angular_dep} to be true, $\det(\mathbf{M}) = 0$, and so
  \begin{gather*}
    (1 - \cos(2\pi\nu))^2 + (\sin(2\pi\nu))^2 = 0 \\
    1 - 2\cos(2\pi\nu) + \cos^2(2\pi\nu) + \sin^2(2\pi\nu) = 0 \\
    2 - 2\cos(2\pi\nu) = 0 \\
    \cos(2\pi\nu) = 1
  \end{gather*}\par
  %
  Hence $\nu = n \in \bb{Z}$ as required. \par
  %
  We have therefore shown that
  \begin{equation*}
    \Theta(\theta) = A \cos(n\theta) + B \sin(n \theta)
  \end{equation*}
\end{proof}

For the radial component, we can rewrite \eqref{eq:ch2_r_dep} as follows:
    \begin{equation}\label{eq:ch2_r_dep_2}
        r^2 \frac{d^2 R}{dr^2} + r \frac{d R}{dr} + (k^2r^2 - \nu^2) R = 0.
    \end{equation}
    \begin{propn}
    Equation \eqref{eq:ch2_r_dep_2} is a Bessel differential equation of order $\nu$.
    \end{propn}
    \begin{proof} Consider the substitution $r=kz$. Then
        \begin{align*}
            \frac{dR}{dr} = \frac{1}{k} \frac{dR}{dz}, ~~
            \frac{d^2R}{dr^2} = \frac{1}{k^2} \frac{d^2R}{dz^2}.
        \end{align*}\par
    %
    So \eqref{eq:ch2_r_dep_2} becomes
        \begin{align}
            \frac{r^2}{k^2}\frac{d^2R}{dz^2}
                + \frac{r}{k}\frac{dR}{dz}
                &+ (k^2r^2 - \nu^2)R = 0, \\
            z^2 \frac{d^2 R}{dz^2}
                + z \frac{dR}{dz}
                &+ (z^2 - \nu^2)R = 0.
        \end{align}
    This fits the defintion of the Bessel differential equation \eqref{eq:bessel_differential}.
    \end{proof}

Since $R(kr)$ satisfies a Bessel differential equation, the Bessel functions are solutions, and by the superposition principle any linear superposition of these is also a solution. We will need to consider the Sommerfeld radiation condition in order specify the general solution for $R(r)$.

\begin{defn}\emph{The Sommerfeld radiation condition.} \parencite{martin06scattering} \label{defn:sommerfeld_radiation_condition}
  \[ r^{1/2} \left( \partialfrac{\Phisc}{r} - ik\Phisc \right) \rightarrow 0 \text{ as } r \rightarrow \infty\]
\end{defn}
Naively, the Sommerfeld radiation condition states that wave diffuses as $r\rightarrow\infty$, and that the scattered waves are not reflecting back and incoming from infinity -- something we wouldn't expect to happen physically. It is therefore reasonable to apply this condition to our problem.

To discuss the behaviour of Hankel functions in relation to the Sommerfeld radiation condition, we first consider their asymptotic expansions \parencite{wong89asymptoticapprox}.
\begin{gather*}
  H^{(1)}_n(kr) \sim \sqrt{\frac{2}{\pi kr}} e^{i(kr - \frac{n\pi}{2} - \frac{\pi}{4})} + O\left(\frac{1}{kr}\right) \text{ as } r \rightarrow \infty \\
  H^{(2)}_n(kr) \sim \sqrt{\frac{2}{\pi kr}} e^{-i(kr - \frac{n\pi}{2} - \frac{\pi}{4})} + O\left(\frac{1}{kr}\right) \text{ as } r \rightarrow \infty \\
\end{gather*}

\begin{propn}
  Hankel functions of the first kind satisfy the Sommerfeld radiation condition. Hankel functions of the second kind do not.
\end{propn}
\begin{proof}
  First consider Hankel functions of the first kind. We now know that
  \begin{equation}
    H_n(kr) \sim \sqrt{\frac{2}{\pi kr}} e^{i(kr - \frac{n\pi}{2} - \frac{\pi}{4})} + O\left(\frac{1}{kr}\right) \text{ as } r \rightarrow \infty.
  \end{equation}

  Then as $r\rightarrow\infty$
  \begin{align*}
    &\partialfrac{H^{(1)}_n(kr)}{r}
    \sim \frac{d}{dr} \left\{
    \sqrt{\frac{2}{k\pi}} r^{-1/2} e^{i(kr - n\pi/2- \pi/4)}
    \right\} + O\left(\frac{1}{kr^2}\right)\\
    &\sim\left\{
    \sqrt{\frac{2}{k\pi}} \frac{d}{dr} (r^{-1/2}) e^{i(kr - n\pi/2- \pi/4)}
    + \sqrt{\frac{2}{k\pi}} r^{-1/2} \frac{d}{dr}(e^{i(kr - n\pi/2- \pi/4)})
    \right\} + O\left(\frac{1}{kr^2}\right)\\
    &\sim \sqrt{\frac{2}{k\pi}} \left\{
    \left(-\frac{1}{2} \right) r^{-3/2} e^{i(kr - n\pi/2- \pi/4)} + r^{-1/2} (ik) e^{i(kr - n\pi/2- \pi/4)}
    \right\} + O\left(\frac{1}{kr^2}\right)\\
    &\sim \sqrt{\frac{2}{kr\pi}} e^{i(kr - n\pi/2- \pi/4)} \left\{
    \left(-\frac{1}{2r} \right) + ik
    \right\}  + O\left(\frac{1}{kr^2}\right).
  \end{align*}

  Hence we have
  \begin{align*}
    r^{1/2} \left(
      \partialfrac{H^{(1)}_n(kr)}{r} - ik H^{(1)}_n(kr)
    \right)
    & \sim \left(-\frac{1}{2r} \right)
      \sqrt{\frac{2}{k\pi}} e^{i(kr - n\pi/2- \pi/4)} \\
    & \sim -\sqrt{\frac{1}{2kr^2\pi}} e^{i(kr - n\pi/2- \pi/4)}\\
  \end{align*}

  This tends to zero as $r\rightarrow\infty$, satisfying the Sommerfeld radiation condition. We can do this same analysis on Hankel functions of the second kind:
  \begin{gather*}
    \partialfrac{H^{(2)}_n(kr)}{r} \sim \sqrt{\frac{2}{kr\pi}} e^{-i(kr - n\pi/2- \pi/4)} \left\{
    -\frac{1}{2r} - ik
    \right\} + O\left(\frac{1}{kr^2}\right), \\
  \end{gather*}
  so
  \begin{gather*}
    r^{1/2} \left(
      \partialfrac{H^{(1)}_n(kr)}{r} - ik H^{(1)}_n(kr)
    \right)
    \sim \sqrt{\frac{2}{kr\pi}} e^{-i(kr - n\pi/2- \pi/4)} \left\{
    -\frac{1}{2r} - 2ik
    \right\}
  \end{gather*}
  which does not satisfy the Sommerfeld radiation condtion.
\end{proof}

From now on we refer to $H^{(1)}_n$ as $H_n$ for simplicity.
%----------------------------------------------------------------
%               General solution
%----------------------------------------------------------------
We can now combine our solutions for $\Theta$ and $R$ to find an expression for $\Phisc$.

\begin{propn}\label{eq:gen_sol_scatterin_outside_cylinder}
  The general solution for scattered field can be expressed as follows.
    \begin{equation} \label{eq:ch2_gen_eq_scattered_field}
        \Phisc = \sum^\infty_{n=0} \epsilon_n i^n B_n H_n(kr) \{A\cos(n\theta) + B \sin(n\theta)\}
    \end{equation}
  for $B_n$ a constant.
\end{propn}
\begin{proof}
  We have already showed that the radial component of the scalar potential must be a Hankel function. We have also showed that the angular component must be of the form
    \begin{align*}
      A \cos(n\theta) + B \sin(n \theta).
    \end{align*}

  We expect our solution to have similar simetries to the incident field, so we can set $B$ to zero. We are then left with a solution of the form
    \begin{equation}
      C_n H_n(kr) \cos(n\theta).
    \end{equation}
  for any constant $C_n$. It will be useful when setting the boundary conditions find $ B_n = C_n/\epsilon_n i^n$ instead of $C_n$. So the general solution is
  \begin{equation}
    \sum^\infty_{n=0} \epsilon_n i^n B_n H_n(kr) \cos(n\theta)
  \end{equation}
  as expected.
\end{proof}

%----------------------------------------------------------------
%           ~ B O U N D A R Y  C O N D I T I O N S ~
%----------------------------------------------------------------

\section{Boundary conditions}

At the beginning of the chapter we defined
  \[ \Phitot = \Phisc + \Phiin \]
so we can express $\Phitot$ as follows, using \eqref{propn:ch2_inc_field_infty_sum}
\begin{equation}
  \sum^\infty_{n=0} \epsilon_n i^n [B_n H_n(kr) + J_n(kr)] \cos(n\theta).
\end{equation}

The only thing left to do now is to find the value of $B_n$.

\subsection*{Neumann boundary condition}

We first cosider the boundary at the cylinder wall to satisfy a Neumann boundary condition.
  \begin{defn}
    \parencite[$\S$1.3.2]{martin06scattering} A boundary is sound-hard if
      \begin{align*}
        \partialfrac{u}{r} = 0 \text{,  on } r = \sigma.
      \end{align*}
  \end{defn} \par
%
Equivalently, we can express this boundary condition in terms of $\Phi$:
  \begin{align*}
    \partialfrac{\Phi}{r}=0 \text{,  on } r = \sigma.
  \end{align*} \par
%
We can now apply this to find an expression for the constant terms in \eqref{eq:ch2_gen_eq_scattered_field}. Differentiating this equation gives
  \begin{equation}
    \sum^\infty_{n=0} \epsilon_n k i^n
    \{ J'_n(kr) + B_n H'_n(kr) \} \cos(n\theta) = 0
  \end{equation}
The expression inside the braces must be zero for each $n$ at the boundary $r=\sigma$. Hence, we get an expression for $B_n$:
  \begin{equation}
    B_n= \frac{J'_n(k\sigma)}{H'_n(k\sigma)}.
  \end{equation}

\subsection*{Dirichlet boundary condition}\label{ss:ch2_dirichlet_bcs}
We now consider the Dirichlet boundary condition.
  \begin{defn}
    \parencite[$\S$1.3.2]{martin06scattering} A body is sound-soft if
      \[
      u = 0 \text{,  on } r = \sigma
      \]
  \end{defn}
Hence, $\Phi_{sc} = - \Phi_{in}$ on $r=\sigma$,
  \begin{align*}
    \sum^\infty_{n=0} \epsilon_n i^n B_n H_n(kr) \cos(n\theta)
    & = - \sum^\infty_{n=0} \epsilon_n i^n J_n(kr) \cos(n\theta) \\
    \text{hence for all } n,~ B_n H_n (k\sigma) &= - J_n (k\sigma) \\
  \end{align*}
Hence the Dirichlet boundary condition specifies $B_n$ as follows.
  \begin{equation}
    B_n = \frac{- J_n(k\sigma)}{H_n(k\sigma)}
  \end{equation}
